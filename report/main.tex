\documentclass[11pt,a4paper]{article}
\usepackage[utf8]{inputenc}
\usepackage[margin=1in]{geometry}
\usepackage{graphicx}
\usepackage{hyperref}
\usepackage{amsmath}
\usepackage{booktabs}
\usepackage{caption}
\usepackage{listings}
\usepackage{xcolor}
\usepackage{float}

% Code listing style
\lstset{
    basicstyle=\ttfamily\small,
    breaklines=true,
    frame=single,
    backgroundcolor=\color{gray!10}
}

% Hyperlink setup
\hypersetup{
    colorlinks=true,
    linkcolor=blue,
    filecolor=magenta,
    urlcolor=cyan,
}

\title{
    \textbf{ClickHouse vs Elasticsearch:} \\
    \large A Performance Comparison for Analytical Workloads
}

\author{
    Karl Seryani (251-304-976) \\
    Arik Dhaliwal \\
    Raghav Gulati \\[0.5cm]
    \textit{COMPSCI 4411/9538} \\
    \textit{Western University}
}

\date{\today}

\begin{document}

\maketitle

\begin{abstract}
This report presents a performance comparison of ClickHouse and Elasticsearch for analytical workloads. Inspired by mpathic's migration from Elasticsearch to ClickHouse Cloud, we benchmarked both systems using healthcare datasets of 1M, 10M, and 100M rows. Our results show that ClickHouse's columnar architecture delivers 20-23x faster data ingestion, 5.8-9.5x better compression, and 10-24x faster query performance on analytical operations. We examine the architectural differences that drive these performance gaps and provide guidance on when each system is appropriate.
\end{abstract}

\tableofcontents
\newpage

\section{Introduction}

\subsection{Background and Motivation}

Choosing the right database system significantly impacts performance and costs, especially at scale. Mpathic, an AI healthcare startup, faced this decision when their Elasticsearch-based infrastructure struggled with analytical workloads. Their data scientists needed to run complex queries joining multiple tables for ML experimentation, but these queries took 15-20 minutes. After migrating to ClickHouse Cloud, the same queries completed in 2-3 minutes \cite{mpathic2024case}.

This case study highlights a fundamental trade-off in database design: systems optimized for search (like Elasticsearch) have different architectures than those optimized for analytics (like ClickHouse). Understanding these differences helps organizations choose the right tool for their workload.

\subsection{Project Objectives}

Our project has three goals:

\textbf{First}, examine the architectural differences between ClickHouse and Elasticsearch. ClickHouse uses columnar storage optimized for reading and aggregating large datasets \cite{clickhouse2024architecture}. Elasticsearch uses an inverted index structure designed for text search and document retrieval \cite{elasticsearch2024docs}.

\textbf{Second}, conduct practical benchmarks on realistic healthcare data. We generated datasets with 1M, 10M, and 100M rows and tested seven query types: simple aggregations, multi-level grouping, time-series analysis, filtered aggregations, joins, complex analytical queries, and concurrent load.

\textbf{Third}, analyze when each system is appropriate. Beyond raw performance, we consider factors like operational complexity, storage costs, and developer experience.

\subsection{Significance}

This study is relevant for organizations dealing with:

\begin{itemize}
    \item Time-series data requiring aggregations over time windows
    \item Large-scale analytics with billions of rows
    \item Complex SQL operations including joins and subqueries
    \item Regulated industries requiring HIPAA or GDPR compliance
    \item Cost-sensitive environments where storage and compute costs matter
\end{itemize}

\section{Background Research}

\subsection{Columnar vs Document Storage}

The core architectural difference lies in how data is stored. ClickHouse stores each column separately, so a query reading two columns from a ten-column table only reads 20\% of the data. Elasticsearch stores complete documents together, requiring full document reads even when queries need just a few fields.

For analytics, columnar storage provides substantial advantages. When running \texttt{SELECT AVG(cost) FROM events WHERE severity='Critical'}, ClickHouse reads only the \texttt{cost} and \texttt{severity} columns. Elasticsearch must read entire documents or use separate columnar structures (doc values), increasing storage overhead.

This architectural choice affects compression too. Columnar storage groups similar values together, enabling better compression. ClickHouse uses compression algorithms like LZ4 and ZSTD, with ZSTD being the default in ClickHouse Cloud \cite{clickhouse2024compression}. Our benchmarks confirmed compression ratios between 5.8x and 9.5x compared to Elasticsearch.

\subsection{Query Processing}

ClickHouse uses vectorized query execution, processing data in batches that leverage CPU SIMD instructions. This approach processes multiple values simultaneously, improving throughput. The system also generates optimized code for each query rather than using a generic interpreter \cite{clickhouse2024architecture}.

Elasticsearch distributes queries across shards, using inverted indexes for fast term lookups. For text search like "find documents containing diabetes", this architecture excels. However, aggregations require iterating through matched documents to accumulate values, which is less efficient than columnar scans.

\subsection{The mpathic Case Study}

Mpathic's migration provides real-world validation. Their challenges with Elasticsearch included:

\begin{itemize}
    \item Slow ML experimentation due to 15-20 minute query times
    \item No native SQL joins, requiring data denormalization or application-side joins
    \item High operational overhead managing EC2-based clusters
    \item Complex compliance setup for HIPAA requirements
\end{itemize}

After migrating to ClickHouse Cloud, they achieved:

\begin{itemize}
    \item 3.75x faster ML pipelines (15 min to 4 min)
    \item 7-10x faster complex join queries
    \item 60\% storage cost reduction through better compression
    \item Simplified HIPAA compliance with managed security features
\end{itemize}

These improvements stem from ClickHouse's columnar architecture and native SQL support \cite{mpathic2024case}.

\section{Methodology}

\subsection{Dataset}

We generated synthetic healthcare data with the following schema:

\begin{itemize}
    \item \texttt{event\_id}: UUID
    \item \texttt{patient\_id}: Integer
    \item \texttt{timestamp}: DateTime
    \item \texttt{department}: String (Cardiology, Emergency, etc.)
    \item \texttt{severity}: Enum (Critical, High, Medium, Low)
    \item \texttt{diagnosis}: Text description
    \item \texttt{cost}: Decimal
    \item \texttt{duration\_minutes}: Integer
\end{itemize}

We created three datasets: 1M rows (small hospital), 10M rows (regional health network), and 100M rows (approaching national scale). This allows observing how performance changes with data volume.

\subsection{Infrastructure}

\textbf{ClickHouse}: Deployed on ClickHouse Cloud with MergeTree engine, LZ4 compression, and time-based partitioning.

\textbf{Elasticsearch}: Deployed on Elasticsearch Cloud with 3 shards, 1 replica, and explicit field type mappings.

Both used managed cloud services to reflect realistic deployment scenarios and eliminate infrastructure variability.

\subsection{Benchmark Queries}

We tested seven query patterns:

\begin{enumerate}
    \item \textbf{Simple Aggregation}: \texttt{SELECT department, COUNT(*), AVG(cost) GROUP BY department}
    \item \textbf{Multi-Level GROUP BY}: Grouping by department and severity
    \item \textbf{Time-Series}: Daily aggregations with date functions
    \item \textbf{Filtered Aggregation}: WHERE + GROUP BY + HAVING clauses
    \item \textbf{JOIN}: Joining patients and medical events tables
    \item \textbf{Complex Query}: Subqueries with multiple aggregations
    \item \textbf{Concurrent Load}: 5 simultaneous queries
\end{enumerate}

Each query ran 5 times after warm-up, and we measured average execution time, min/max range, and standard deviation.

\section{Results}

\subsection{Data Ingestion}

\begin{table}[H]
\centering
\caption{Data Loading Performance}
\begin{tabular}{lrrr}
\toprule
\textbf{Dataset} & \textbf{ClickHouse} & \textbf{Elasticsearch} & \textbf{Speedup} \\
\midrule
1M rows & 7.1s & 145.2s & 20.5x \\
10M rows & 65.6s & 1534.0s & 23.4x \\
100M rows & 650.9s & 15222s & 23.4x \\
\bottomrule
\end{tabular}
\end{table}

ClickHouse maintained consistent throughput around 150K rows/second across all dataset sizes. The 100M dataset loaded in under 11 minutes. Elasticsearch achieved about 7K rows/second, requiring over 4 hours for 100M rows.

This difference comes from Elasticsearch's indexing overhead. For each document, it must parse text fields, update inverted indexes, write doc values, store the source document, and update shard metadata. ClickHouse simply appends data to compressed column files and performs merging in the background.

\subsection{Storage Efficiency}

\begin{table}[H]
\centering
\caption{Storage Comparison}
\begin{tabular}{lrrr}
\toprule
\textbf{Dataset} & \textbf{ClickHouse} & \textbf{Elasticsearch} & \textbf{Ratio} \\
\midrule
1M rows & 13.36 MB & 77.63 MB & 5.8x \\
10M rows & 136.54 MB & 1191.14 MB & 8.7x \\
100M rows & 1365.4 MB & 13009.05 MB & 9.5x \\
\bottomrule
\end{tabular}
\end{table}

The compression advantage increased with dataset size, reaching 9.5x at 100M rows. For 100M rows, ClickHouse used 1.33 GB while Elasticsearch used 12.7 GB. At cloud storage prices around \$0.10/GB/month, this translates to substantial cost savings at scale.

\subsection{Query Performance}

\begin{table}[H]
\centering
\caption{Query Performance Summary (100M row dataset)}
\begin{tabular}{lrrr}
\toprule
\textbf{Query Type} & \textbf{ClickHouse} & \textbf{Elasticsearch} & \textbf{Speedup} \\
\midrule
Simple Aggregation & 1.52s & 18.76s & 12.3x \\
Multi-Level GROUP BY & 1.69s & 21.46s & 12.7x \\
Time-Series & 1.42s & 31.89s & 22.4x \\
Filtered Aggregation & 0.76s & 12.57s & 16.6x \\
JOIN & 2.88s & 68.95s & 24.0x \\
Complex Query & 2.57s & 39.88s & 15.5x \\
Concurrent (5 queries) & 2.13s & 24.57s & 11.5x \\
\bottomrule
\end{tabular}
\end{table}

Key findings:

\textbf{Time-series queries} showed ClickHouse's largest advantage (22.4x) due to time-based partitioning and efficient date functions.

\textbf{JOIN performance} was dramatically better (24x) because Elasticsearch doesn't support native SQL joins. Our Elasticsearch benchmark used application-side joins requiring thousands of individual queries.

\textbf{Simple aggregations} showed consistent 10-15x advantages. The performance gap widened as datasets grew, indicating that ClickHouse's architecture scales better.

\textbf{Concurrent load} maintained similar speedup ratios, showing both systems handle parallelism well, but ClickHouse's base performance advantage persists.

Across all benchmarks, ClickHouse demonstrated:
\begin{itemize}
    \item Average 10-15x speedup on typical aggregations
    \item Up to 24x faster on joins at scale
    \item Consistent performance as datasets grow
    \item Low variance (standard deviation under 5\%)
\end{itemize}

\subsection{Validation Against mpathic's Results}

Our benchmarks align with mpathic's reported improvements:

\begin{itemize}
    \item \textbf{Ingestion}: Our 20-23x speedup matches their 4x faster data refresh cycles
    \item \textbf{Storage}: Our 9.5x compression ratio validates their 60\% cost reduction
    \item \textbf{Joins}: Our 24x improvement confirms their 7-10x faster complex queries
    \item \textbf{Scalability}: Consistent performance from 1M to 100M rows supports their success at billion-row scale
\end{itemize}

\section{Discussion}

\subsection{Why These Differences Matter}

The performance gaps we measured have real business impact:

\textbf{Development speed}: A query taking 30 seconds vs 2 seconds changes how data scientists work. Faster iteration enables more experimentation and better models.

\textbf{Infrastructure costs}: 20x faster ingestion means you need 5\% of the compute capacity. 9x better compression reduces storage bills proportionally.

\textbf{Real-time analytics}: Queries completing in seconds rather than minutes enable real-time dashboards and operational analytics.

\textbf{Architectural simplicity}: Native SQL joins eliminate the need for data denormalization or complex application logic.

\subsection{When to Choose ClickHouse}

ClickHouse fits best when:

\textbf{Workload characteristics}:
\begin{itemize}
    \item Analytical queries dominate (aggregations, GROUP BY)
    \item Time-series analysis is important
    \item Queries need joins across tables
    \item Data is mostly appended, rarely updated
\end{itemize}

\textbf{Technical requirements}:
\begin{itemize}
    \item SQL compatibility matters
    \item Storage costs must be minimized
    \item Query performance is critical
    \item Data is structured or semi-structured
\end{itemize}

\subsection{When to Choose Elasticsearch}

Elasticsearch remains superior for:

\textbf{Workload characteristics}:
\begin{itemize}
    \item Full-text search is the primary use case
    \item Document retrieval by ID is common
    \item Individual documents need frequent updates
    \item Relevance scoring matters
\end{itemize}

\textbf{Technical requirements}:
\begin{itemize}
    \item Log analytics and monitoring
    \item Search-as-you-type interfaces
    \item Geographic queries
    \item Text analysis (stemming, synonyms)
\end{itemize}

\subsection{Limitations}

Our study has limitations:

\begin{itemize}
    \item Focused only on analytical queries, not search
    \item Synthetic data may not capture all production patterns
    \item Used default configurations without extensive tuning
    \item Single-node deployments, not distributed clusters
    \item Did not measure actual cloud costs
\end{itemize}

Future work could address these through larger-scale tests, distributed deployments, and workloads including text search where Elasticsearch excels.

\section{Conclusion}

Database architecture fundamentally determines performance characteristics. Our benchmarks demonstrate that ClickHouse's columnar architecture provides substantial advantages for analytical workloads:

\begin{itemize}
    \item 20-23x faster data ingestion
    \item 5.8-9.5x better compression
    \item 10-24x faster analytical queries
    \item Native SQL support for complex operations
\end{itemize}

These aren't marginal improvements. They transform what's possible: queries that took minutes complete in seconds, data loads that took hours finish in minutes, and storage costs drop by an order of magnitude.

The mpathic case study validates these findings in production. Their migration resulted in measurable business impacts: faster ML experimentation, reduced costs, and simplified operations.

However, this isn't a universal recommendation. Elasticsearch excels at full-text search and document retrieval. The right choice depends on your workload. For analytical processing of structured data, ClickHouse's architecture provides clear advantages. For search applications, Elasticsearch's inverted index design is superior.

The key insight: architecture matters more than optimization. You cannot tune a row-store to match columnar performance on analytics, just as you cannot make a columnar system match inverted index performance on text search. Understanding these trade-offs enables informed decisions that align technical capabilities with business requirements.

For healthcare organizations like mpathic, financial services firms, IoT platforms, and other analytical use cases, ClickHouse offers compelling advantages. For search applications, log analysis, and document-centric workloads, Elasticsearch remains the better choice.

\section*{Acknowledgments}

We thank the ClickHouse and Elasticsearch communities for their documentation and support. The mpathic case study provided valuable real-world context. We also acknowledge Western University for supporting this research.

\begin{thebibliography}{9}

\bibitem{clickhouse2024compression}
ClickHouse Inc. (2024).
\textit{ClickHouse Documentation: Compression in ClickHouse.}
Retrieved from \url{https://clickhouse.com/docs/en/data-compression/compression-in-clickhouse}

\bibitem{clickhouse2024architecture}
ClickHouse Inc. (2024).
\textit{ClickHouse Documentation: Architecture Overview.}
Retrieved from \url{https://clickhouse.com/docs/development/architecture}

\bibitem{mpathic2024case}
ClickHouse Inc. (2024).
\textit{How mpathic built better ML workflows by switching from Elasticsearch to ClickHouse Cloud.}
Retrieved from \url{https://clickhouse.com/blog/mpathic-better-ml-elastic-to-clickhouse-migration}

\bibitem{elasticsearch2024docs}
Elastic N.V. (2024).
\textit{Elasticsearch Guide [8.x].}
Retrieved from \url{https://www.elastic.co/guide/en/elasticsearch/reference/current/index.html}

\end{thebibliography}

\end{document}
